% LaTeX Template For MATH 490 @ VCU
\documentclass[11pt]{article}

\usepackage{hyperref}
\usepackage{amsmath}
\usepackage{amsthm}
\usepackage{amssymb}
\usepackage{enumerate}
\usepackage{enumitem}
\usepackage{titlesec}
\usepackage{multicol}
\usepackage{multirow}
\usepackage{mathtools}
\usepackage{mdframed}
\usepackage{tocloft}
\usepackage{tcolorbox}
\usepackage{extarrows}

\setlist{nosep}
% \setlist[enumerate]{label=(\alph*)}

\renewcommand{\arraystretch}{0.75}

\definecolor{defcolor}{RGB}{255,236,236}    % light red
\definecolor{ngtcolor}{RGB}{255,242,242}    % lighter red
\definecolor{lnkcolor}{RGB}{0,0,180}        % blue
\definecolor{thmcolor}{RGB}{236,236,255}    % light blue
\definecolor{lemcolor}{RGB}{239,239,255}    % lighter blue
\definecolor{procolor}{RGB}{242,242,255}    % lighter lighter blue
\definecolor{crlcolor}{RGB}{245,245,255}    % lighter lighter lighter blue
\definecolor{xmpcolor}{RGB}{255,240,225}    % light orange
\definecolor{rmkcolor}{RGB}{233,255,235}    % light green
\definecolor{axicolor}{RGB}{255,255,233}    % light yellow
\definecolor{notcolor}{RGB}{255,255,244}    % lighter yellow
\definecolor{whacolor}{RGB}{250,250,250}    % lighter gray
\definecolor{reccolor}{RGB}{255,244,255}    % lighter purple

\hypersetup{
    colorlinks,
    citecolor=lnkcolor,
    filecolor=lnkcolor,
    linkcolor=lnkcolor,
    urlcolor=lnkcolor
}

\newtheoremstyle{break}
    {\topsep/1.5} % space above
    {\topsep/2.2} % space below
    {}          % body font
    {}          % indent amount
    {\rmfamily} % theorem head font
    {.}          % punctuation after theorem head
    {\newline}  % space after theorem head
    {\textbf{\thmname{#1}\thmnumber{ #2}}\thmnote{\text{ (#3)}}}
                % theorem hed spec. (empty = "normal")

\theoremstyle{break}
\newmdtheoremenv[
    backgroundcolor=thmcolor,
    linecolor=black,
    linewidth=1pt,
    topline=true,
    bottomline=true,
    rightline=true,
    skipabove=\topsep/1.5,
    skipbelow=\topsep/2.2
]{theorem}{Theorem}[section]
\newmdtheoremenv[
    backgroundcolor=crlcolor,
    linecolor=black,
    linewidth=1pt,
    topline=true,
    bottomline=true,
    rightline=true,
    skipabove=\topsep/1.5,
    skipbelow=\topsep/2.2
]{corollary}[theorem]{Corollary}
\newmdtheoremenv[
    backgroundcolor=lemcolor,
    linecolor=black,
    linewidth=1pt,
    topline=true,
    bottomline=true,
    rightline=true,
    skipabove=\topsep/1.5,
    skipbelow=\topsep/2.2
]{lemma}[theorem]{Lemma}
\newmdtheoremenv[
    backgroundcolor=axicolor,
    linecolor=black,
    linewidth=1pt,
    topline=true,
    bottomline=true,
    rightline=true,
    skipabove=\topsep/1.5,
    skipbelow=\topsep/2.2
]{axiom}[theorem]{Axiom}
\newmdtheoremenv[
    backgroundcolor=procolor,
    linecolor=black,
    linewidth=1pt,
    topline=true,
    bottomline=true,
    rightline=true,
    skipabove=\topsep/1.5,
    skipbelow=\topsep/2.2
]{proposition}[theorem]{Proposition}
\newmdtheoremenv[
    backgroundcolor=notcolor,
    linecolor=black,
    linewidth=1pt,
    topline=true,
    bottomline=true,
    rightline=true,
    skipabove=\topsep/1.5,
    skipbelow=\topsep/2.2
]{notation}[theorem]{Notation}
\newmdtheoremenv[
    backgroundcolor=defcolor,
    linecolor=black,
    linewidth=1pt,
    topline=true,
    bottomline=true,
    rightline=true,
    skipabove=\topsep/1.5,
    skipbelow=\topsep/2.2
]{definition}[theorem]{Definition}
\newmdtheoremenv[
    backgroundcolor=ngtcolor,
    linecolor=black,
    linewidth=1pt,
    topline=true,
    bottomline=true,
    rightline=true,
    skipabove=\topsep/1.5,
    skipbelow=\topsep/2.2
]{negation}[theorem]{Negation}
\newmdtheoremenv[
    backgroundcolor=rmkcolor,
    linecolor=black,
    linewidth=1pt,
    topline=true,
    bottomline=true,
    rightline=true,
    skipabove=\topsep/1.5,
    skipbelow=\topsep/2.2
]{remark}[theorem]{Remark}
\newmdtheoremenv[
    backgroundcolor=xmpcolor,
    linecolor=black,
    linewidth=1pt,
    topline=true,
    bottomline=true,
    rightline=true,
    skipabove=\topsep/1.5,
    skipbelow=\topsep/2.2
]{example}[theorem]{Example}
\newmdtheoremenv[
    backgroundcolor=whacolor,
    linecolor=black,
    linewidth=1pt,
    topline=true,
    bottomline=true,
    rightline=true,
    skipabove=\topsep/1.5,
    skipbelow=\topsep/2.2
]{problem}[theorem]{Problem}
\newmdtheoremenv[
    backgroundcolor=whacolor,
    linecolor=black,
    linewidth=1pt,
    topline=true,
    bottomline=true,
    rightline=true,
    skipabove=\topsep/1.5,
    skipbelow=\topsep/2.2
]{question}[theorem]{Question}
\newmdtheoremenv[
    backgroundcolor=reccolor,
    linecolor=black,
    linewidth=1pt,
    topline=true,
    bottomline=true,
    rightline=true,
    skipabove=\topsep/1.5,
    skipbelow=\topsep/2.2
]{recall}[theorem]{Recall}

\DeclareMathOperator{\arcsec}{arcsec}
\DeclareMathOperator{\arccot}{arccot}
\DeclareMathOperator{\arccsc}{arccsc}
\DeclareMathOperator{\interior}{int}
\DeclareMathOperator{\closure}{cl}
\DeclareMathOperator{\boundary}{bd}

\newcommand{\differentiate}[1]{\dfrac{\dd}{\dd{#1}}}
\newcommand{\pdifferentiate}[1]{\dfrac{\partial}{\partial {#1}}}
\newcommand{\derivative}[2]{\dfrac{\dd{#1}}{\dd{#2}}}
\newcommand{\scndderivative}{D^2\!\,}
\newcommand{\highderivative}[1]{D^{#1}\!\,}
\newcommand{\dirderivative}[1]{D_{#1}\!\,}
\newcommand{\pderivative}[2]{\dfrac{\partial {#1}}{\partial {#2}}}
\newcommand{\scndpderivative}[3]{\dfrac{\partial^2 {#1}}{\partial {#3}\partial {#2}}}
\newcommand{\highpderivative}[4]{\dfrac{\partial^{#2} {#1}}{\partial{#4}\cdots\partial{#3}}}
\newcommand{\dd}{\text{d}}
\newcommand{\ddi}{\text{$\,$d}}
\newcommand{\qqed}{{\hfill$\blacksquare$}}
\newcommand{\defeq}{\overset{\text{def}}{=}}
\newcommand{\transpose}{\text{T}}
\newcommand{\bbR}{\mathbb{R}}
\newcommand{\bbN}{\mathbb{N}}
\newcommand{\calL}{\mathcal{L}}
\newcommand{\bfa}{\textbf{a}}
\newcommand{\bfc}{\textbf{c}}
\newcommand{\bfe}{\textbf{e}}
\newcommand{\bff}{\textbf{f}}
\newcommand{\bfg}{\textbf{g}}
\newcommand{\bfh}{\textbf{h}}
\newcommand{\bfp}{\textbf{p}}
\newcommand{\bfr}{\textbf{r}}
\newcommand{\bfv}{\textbf{v}}
\newcommand{\bfu}{\textbf{u}}
\newcommand{\bfx}{\textbf{x}}
\newcommand{\bfy}{\textbf{y}}
\newcommand{\exercise}{This is an exercise left to the reader.}


\linespread{1.9}
\setlength{\textwidth}{6.9in}
\setlength{\textheight}{9.2in}
\setlength{\oddsidemargin}{-0.2in}
\setlength{\evensidemargin}{-0.2in}
\setlength{\topmargin}{-0.2in}
\setlength{\headheight}{0in}
\setlength{\headsep}{0in}
\setlength{\footskip}{0.5in}
\setlength{\multicolsep}{6.2pt}
\setlength{\belowdisplayskip}{0pt}
%\setlength{\belowdisplayshortskip}{0pt}
\setlength{\abovedisplayskip}{0pt}
%\setlength{\abovedisplayshortskip}{0pt}

\setcounter{section}{1}
\numberwithin{equation}{theorem}

\makeatletter
\newcommand{\vast}{\bBigg@{4}}
\newcommand{\Vast}{\bBigg@{5}}
\makeatother

\newcommand*\samethanks[1][\value{footnote}]{\footnotemark[#1]}

\title{\textbf{Calculus A II One-to-One Tutoring}}
\author{Chang, Yung-Hsuan}

\begin{document}
\maketitle

\begin{question}[Some Basic Derivatives]
    Find the derivative with respect to $x$ for the following functions: \vspace{-1.2em}
    \begin{multicols}{4}
        \begin{enumerate}
            \item $y=x^n$;
            \item $y=e^x$;
            \item $y=a^x$;
            \item $y=\ln x$;
            \item $y=\log_a x$;
            \item $y=\sin x$;
            \item $y=\cos x$; and
            \item $y=\tan x$;
        \end{enumerate}
    \end{multicols}
    \vspace{0.01em}
\end{question}
\newpage

\begin{question}[Utilizing the Chain Rule]
    Find the derivative of the following functions: \vspace{-1.4em}
    \begin{multicols}{2}
        \begin{enumerate}
            \item $y=\sqrt[3]{e^x+1}$;
            \item $y=e^{\tan\theta}$;
            \item $y=\sin\left(\dfrac{e^x}{1+e^x}\right)$;
            \item $y=t\sin\left(\pi t\right)$;
            \item $y=\sin(\ln x)$;
            \item $y=\ln\left(\dfrac{x^a}{b^x}\right)$;
            \item $y=\dfrac{1}{\ln x}$;
            \item $y=\ln\left(\left(\sin x\right)^2\right)$; and
            \item $y=\dfrac{\ln x}{1+\ln x}$.
        \end{enumerate}
    \end{multicols}
    \vspace{0.01em}
\end{question}
\newpage

\begin{question}[Applying the Chain Rule]
    Use the fact that $|x|=\sqrt{x^2}$ to find $\differentiate{x}\left(|x|\right)$.
\end{question}
\newpage

\begin{question}[Applying the Chain Rule]
    Find the derivative of $y=x^x$. (Hint: take log to both sides.)
\end{question}
\newpage

\begin{question}[Comprehensive Applications]
    Find the derivative of the following functions:
    \begin{enumerate}
        \item $y=\sqrt[4]{x\sqrt[3]{x\sqrt{x}}}$;
        \item $y=(x-3)\sqrt{x^2+2x+3}$;
        \item $y=x^{\left(\ln x\right)^{111}}$;
        \item $y=\cos(\sin 3x)$;
        \item $y=e^t(1+te^t)$;
        \item $y=x^3e^x$;
        \item $y=\dfrac{x}{e^x}$; and
        \item $y=\dfrac{e^x}{1-e^x}$.
    \end{enumerate}
\end{question}
\newpage

\begin{example}
    Compute $\differentiate{x}\left(\sin\left(\cos 5x\right)\right)$.
\end{example}
\vspace{8em}

\begin{example}
    Compute $\differentiate{x}\left(\sin\left(3\cos x\right)\right)$.
\end{example}
\vspace{8em}

\begin{example}
    Compute $\differentiate{x}\left(\sin\left(\cos bx\right)\right)$ for $b\in\bbR^+$.
\end{example}
\vspace{8em}

\begin{example}
    Compute $\differentiate{x}\left(\sin\left(a\cos x\right)\right)$ for $a\in\bbR^+$.
\end{example}
\vspace{8em}

\begin{example}
    Compute $\differentiate{x}\left(\sin\left(a\cos bx\right)\right)$ for $a, b\in\bbR^+$.
\end{example}
\vspace{8em}

\begin{example}
    Compute $\pderivative{w}{x}$ for $w=\sin\left(y\cos x\right)$.
\end{example}
\vspace{8em}

\begin{example}
    Compute $\pderivative{w}{x}$ for $w=z\sin\left(\cos x\right)$.
\end{example}
\vspace{8em}

\begin{example}
    Compute $\pderivative{w}{x}$ for $w=z\sin\left(y\cos x\right)$.
\end{example}
\vspace{8em}

\begin{example}
    Find the first derivatives of the following functions:
    \begin{enumerate}
        \item $w=x^4+5xy^3$;
        \item $w=x^2y-3y^4$;
        \item $w=x^3\sin y$;
        \item $w=e^{xt}$;
        \item $w=\ln(x+t^2)$;
        \item $w=\dfrac{e^x}{u+v^2}$;
        \item $w=x^y$; and
        \item $w=\ln(x+2y+3z)$.
    \end{enumerate}
\end{example}

\end{document}