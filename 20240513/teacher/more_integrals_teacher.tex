% LaTeX Template For MATH 490 @ VCU
\documentclass[11pt]{article}

\usepackage{hyperref}
\usepackage{amsmath}
\usepackage{amsthm}
\usepackage{amssymb}
\usepackage{enumerate}
\usepackage{enumitem}
\usepackage{titlesec}
\usepackage{multicol}
\usepackage{multirow}
\usepackage{mathtools}
\usepackage{mdframed}
\usepackage{tocloft}
\usepackage{tcolorbox}
\usepackage{extarrows}

\setlist{nosep}
% \setlist[enumerate]{label=(\alph*)}

\renewcommand{\arraystretch}{1}

\definecolor{defcolor}{RGB}{255,236,236}    % light red
\definecolor{ngtcolor}{RGB}{255,242,242}    % lighter red
\definecolor{lnkcolor}{RGB}{0,0,180}        % blue
\definecolor{thmcolor}{RGB}{236,236,255}    % light blue
\definecolor{lemcolor}{RGB}{239,239,255}    % lighter blue
\definecolor{procolor}{RGB}{242,242,255}    % lighter lighter blue
\definecolor{crlcolor}{RGB}{245,245,255}    % lighter lighter lighter blue
\definecolor{xmpcolor}{RGB}{255,240,225}    % light orange
\definecolor{rmkcolor}{RGB}{233,255,235}    % light green
\definecolor{axicolor}{RGB}{255,255,233}    % light yellow
\definecolor{notcolor}{RGB}{255,255,244}    % lighter yellow
\definecolor{whacolor}{RGB}{250,250,250}    % lighter gray
\definecolor{reccolor}{RGB}{255,244,255}    % lighter purple

\hypersetup{
    colorlinks,
    citecolor=lnkcolor,
    filecolor=lnkcolor,
    linkcolor=lnkcolor,
    urlcolor=lnkcolor
}

\newtheoremstyle{break}
    {\topsep/1.5} % space above
    {\topsep/2.2} % space below
    {}          % body font
    {}          % indent amount
    {\rmfamily} % theorem head font
    {.}          % punctuation after theorem head
    {\newline}  % space after theorem head
    {\textbf{\thmname{#1}\thmnumber{ #2}}\thmnote{\text{ (#3)}}}
                % theorem hed spec. (empty = "normal")

\newtheoremstyle{no_label}
    {\topsep/1.5} % space above
    {\topsep/2.2} % space below
    {}          % body font
    {}          % indent amount
    {\rmfamily} % theorem head font
    {.}          % punctuation after theorem head
    {\newline}  % space after theorem head
    {\textbf{\thmname{#1}\thmnumber{}}\thmnote{\text{ (#3)}}}
                % theorem hed spec. (empty = "normal")

\theoremstyle{break}
\newmdtheoremenv[
    backgroundcolor=thmcolor,
    linecolor=black,
    linewidth=1pt,
    topline=true,
    bottomline=true,
    rightline=true,
    skipabove=\topsep/1.5,
    skipbelow=\topsep/2.2
]{theorem}{Theorem}[section]
\newmdtheoremenv[
    backgroundcolor=crlcolor,
    linecolor=black,
    linewidth=1pt,
    topline=true,
    bottomline=true,
    rightline=true,
    skipabove=\topsep/1.5,
    skipbelow=\topsep/2.2
]{corollary}[theorem]{Corollary}
\newmdtheoremenv[
    backgroundcolor=lemcolor,
    linecolor=black,
    linewidth=1pt,
    topline=true,
    bottomline=true,
    rightline=true,
    skipabove=\topsep/1.5,
    skipbelow=\topsep/2.2
]{lemma}[theorem]{Lemma}
\newmdtheoremenv[
    backgroundcolor=axicolor,
    linecolor=black,
    linewidth=1pt,
    topline=true,
    bottomline=true,
    rightline=true,
    skipabove=\topsep/1.5,
    skipbelow=\topsep/2.2
]{axiom}[theorem]{Axiom}
\newmdtheoremenv[
    backgroundcolor=procolor,
    linecolor=black,
    linewidth=1pt,
    topline=true,
    bottomline=true,
    rightline=true,
    skipabove=\topsep/1.5,
    skipbelow=\topsep/2.2
]{proposition}[theorem]{Proposition}
\newmdtheoremenv[
    backgroundcolor=notcolor,
    linecolor=black,
    linewidth=1pt,
    topline=true,
    bottomline=true,
    rightline=true,
    skipabove=\topsep/1.5,
    skipbelow=\topsep/2.2
]{notation}[theorem]{Notation}
\newmdtheoremenv[
    backgroundcolor=defcolor,
    linecolor=black,
    linewidth=1pt,
    topline=true,
    bottomline=true,
    rightline=true,
    skipabove=\topsep/1.5,
    skipbelow=\topsep/2.2
]{definition}[theorem]{Definition}
\newmdtheoremenv[
    backgroundcolor=ngtcolor,
    linecolor=black,
    linewidth=1pt,
    topline=true,
    bottomline=true,
    rightline=true,
    skipabove=\topsep/1.5,
    skipbelow=\topsep/2.2
]{negation}[theorem]{Negation}
\newmdtheoremenv[
    backgroundcolor=rmkcolor,
    linecolor=black,
    linewidth=1pt,
    topline=true,
    bottomline=true,
    rightline=true,
    skipabove=\topsep/1.5,
    skipbelow=\topsep/2.2
]{remark}[theorem]{Remark}
\newmdtheoremenv[
    backgroundcolor=xmpcolor,
    linecolor=black,
    linewidth=1pt,
    topline=true,
    bottomline=true,
    rightline=true,
    skipabove=\topsep/1.5,
    skipbelow=\topsep/2.2
]{example}[theorem]{Example}
\newmdtheoremenv[
    backgroundcolor=whacolor,
    linecolor=black,
    linewidth=1pt,
    topline=true,
    bottomline=true,
    rightline=true,
    skipabove=\topsep/1.5,
    skipbelow=\topsep/2.2
]{problem}[theorem]{Problem}
\newmdtheoremenv[
    backgroundcolor=whacolor,
    linecolor=black,
    linewidth=1pt,
    topline=true,
    bottomline=true,
    rightline=true,
    skipabove=\topsep/1.5,
    skipbelow=\topsep/2.2
]{exercise}[theorem]{Exercise}

\theoremstyle{no_label}
\newmdtheoremenv[
    backgroundcolor=whacolor,
    linecolor=black,
    linewidth=1pt,
    topline=true,
    bottomline=true,
    rightline=true,
    skipabove=\topsep/1.5,
    skipbelow=\topsep/2.2
]{question}{Question}
\newmdtheoremenv[
    backgroundcolor=reccolor,
    linecolor=black,
    linewidth=1pt,
    topline=true,
    bottomline=true,
    rightline=true,
    skipabove=\topsep/1.5,
    skipbelow=\topsep/2.2
]{recall}{Recall}

\DeclareMathOperator{\arcsec}{arcsec}
\DeclareMathOperator{\arccot}{arccot}
\DeclareMathOperator{\arccsc}{arccsc}
\DeclareMathOperator{\interior}{int}
\DeclareMathOperator{\closure}{cl}
\DeclareMathOperator{\boundary}{bd}

\newcommand{\differentiate}[1]{\dfrac{\dd}{\dd{#1}}}
\newcommand{\pdifferentiate}[1]{\dfrac{\partial}{\partial {#1}}}
\newcommand{\derivative}[2]{\dfrac{\dd{#1}}{\dd{#2}}}
\newcommand{\scndderivative}{D^2\!\,}
\newcommand{\highderivative}[1]{D^{#1}\!\,}
\newcommand{\dirderivative}[1]{D_{#1}\!\,}
\newcommand{\pderivative}[2]{\dfrac{\partial {#1}}{\partial {#2}}}
\newcommand{\scndpderivative}[3]{\dfrac{\partial^2 {#1}}{\partial {#3}\partial {#2}}}
\newcommand{\highpderivative}[4]{\dfrac{\partial^{#2} {#1}}{\partial{#4}\cdots\partial{#3}}}
\newcommand{\dd}{\text{d}}
\newcommand{\ddi}{\text{$\,$d}}
\newcommand{\qqed}{{\hfill$\blacksquare$}}
\newcommand{\defeq}{\overset{\text{def}}{=}}
\newcommand{\transpose}{\text{T}}
\newcommand{\bbR}{\mathbb{R}}
\newcommand{\bbN}{\mathbb{N}}
\newcommand{\calL}{\mathcal{L}}
\newcommand{\bfa}{\textbf{a}}
\newcommand{\bfc}{\textbf{c}}
\newcommand{\bfe}{\textbf{e}}
\newcommand{\bff}{\textbf{f}}
\newcommand{\bfg}{\textbf{g}}
\newcommand{\bfh}{\textbf{h}}
\newcommand{\bfp}{\textbf{p}}
\newcommand{\bfr}{\textbf{r}}
\newcommand{\bfv}{\textbf{v}}
\newcommand{\bfu}{\textbf{u}}
\newcommand{\bfx}{\textbf{x}}
\newcommand{\bfy}{\textbf{y}}


\linespread{2}
\setlength{\textwidth}{6.9in}
\setlength{\textheight}{9.2in}
\setlength{\oddsidemargin}{-0.2in}
\setlength{\evensidemargin}{-0.2in}
\setlength{\topmargin}{-0.2in}
\setlength{\headheight}{0in}
\setlength{\headsep}{0in}
\setlength{\footskip}{0.5in}
\setlength{\multicolsep}{6.2pt}
\setlength{\belowdisplayskip}{0pt}
%\setlength{\belowdisplayshortskip}{0pt}
\setlength{\abovedisplayskip}{0pt}
%\setlength{\abovedisplayshortskip}{0pt}

\setcounter{section}{5}
\numberwithin{equation}{theorem}

\makeatletter
\newcommand{\vast}{\bBigg@{4}}
\newcommand{\Vast}{\bBigg@{5}}
\makeatother

\newcommand*\samethanks[1][\value{footnote}]{\footnotemark[#1]}

\title{\textbf{Calculus A (II) One-to-One Tutoring}\\ \Large Practices on Integrals and Fibini's Theorem}
\author{Chang, Yung-Hsuan}

\begin{document}
\maketitle

\begin{example}
    Evaluate the following indefinite integral: $\displaystyle\int \dfrac{1}{1-x^2}\ddi x.$
\end{example}


\begin{example}
    Evalutate the following indefinite integral: $\displaystyle\int \dfrac{1}{a^2+x^2}\ddi x.$
\end{example}


\begin{example}
    Evaluate the following indefinite integral: $\displaystyle\int \dfrac{x^5}{x^3+1}\ddi x.$
\end{example}


\begin{example}
    Evaluate the following indefinite integral: $\displaystyle\int \dfrac{x^2+1}{x}\ddi x.$
\end{example}


\begin{example}
    Evaluate the following indefinite integral: $\displaystyle\int \cos x\cdot e^{\sin x}\ddi x.$
\end{example}


\begin{example}
    Evaluate the following indefinite integral: $\displaystyle\int\dfrac{\sin(\ln t)}{t}\ddi t.$
\end{example}


\begin{example}
    Evaluate the following indefinite integral: $\displaystyle\int x(\ln x)^2\ddi x.$
\end{example}


\begin{example}
    Evaluate the following indefinite integral: $\displaystyle\int x\sin x\ddi x.$
\end{example}


\begin{example}
    Evaluate the following indefinite integral: $\displaystyle\int x\ln x\ddi x.$
\end{example}


\begin{example}
    Evaluate the following indefinite integral: $\displaystyle\int \sin x\cos x\ln(\cos x)\ddi x.$
\end{example}


\begin{example}
    Evaluate the following indefinite integral: $\displaystyle\int e^x\cos x\ddi x.$
\end{example}


\begin{example}
    Evaluate $\displaystyle\int_{0}^{\pi/2}\int_{0}^{\pi/2}\sin x\cos y\ddi x\ddi y.$
\end{example}


\begin{example}
    Evaluate $\displaystyle\int_{1}^{2}\int_{0}^{\pi}y\sin(xy)\ddi y\ddi x.$
\end{example}


\begin{example}
    Evaluate $\displaystyle\iint_D 2y \ddi A,$ where $D$ is the region bounded by the line $y=x$ and the parabola $y=3x-x^2$.
\end{example}


\begin{example}
    Find the volume of the solid that lies under the paraboloid $z=x^2+y^2$ and above the region $D$ in the $xy$-plane bounded by the line $y=2x$ and the parabola $y=x^2$.
\end{example}


\begin{example}
    Evaluate $\displaystyle\iint_D xy\ddi A,$ where $D$ is the region bounded by the line $y=x-1$ and the parabola $y^2=2x+6$.
\end{example}


\begin{example}
    Evaluate $\displaystyle\int_{0}^{4}\int_{0}^{\sqrt{x}}\dfrac{y}{x^2+1}\ddi y\ddi x.$
\end{example}


\begin{example}
    Evaluate $\displaystyle\int_{0}^{3}\int_{0}^{y}e^{-y^2}\ddi x\ddi y.$
\end{example}


\begin{example}
    Evaluate $\displaystyle\int_{0}^{1}\int_{1}^{x}\sin(y^2)\ddi y\ddi x.$
\end{example}


\begin{example}
    Evaluate $\displaystyle\int_{0}^{1}\int_{3y}^{3}e^{x^2}\ddi x\ddi y.$
\end{example}


\begin{example}
    Evaluate $\displaystyle\int_{0}^{1}\int_{y}^{1}\dfrac{\sin x}{x}\ddi x\ddi y.$
\end{example}


\begin{question}
    How can we use the substitution rule for multiple integrals?
\end{question}

\begin{definition}
    The Jacobian of the transformation $T$ given by $x=g(u, v)$ and $y=h(u, v)$ is $$\dfrac{\partial(x, y)}{\partial(u, v)}\coloneq\begin{vmatrix}
        \pderivative{x}{u} & \pderivative{x}{v} \\ \pderivative{y}{u} & \pderivative{y}{v}
    \end{vmatrix}=\pderivative{x}{u}\pderivative{y}{v}-\pderivative{x}{v}\pderivative{y}{u}.$$
\end{definition}

\begin{theorem}[Change of Variables in a Double Integral]\label{cv di}
    Suppose that $T$ is a $C^1$ transformation whose Jacobian is nonzero and that $T$ maps a region $S$ in the $uv$-plane onto a region $R$ in the $xy$-plane. Suppose that $f$ is continuous on $R$ and that $R$ and $S$ are type I or type II plane regions. Suppose also that $T$ is one-to-one, except perhaps on the boundary of $S$. Then $$\displaystyle\iint_R f(x, y)\ddi A=\iint_S f(x(u,v), y(u, v))\left|\dfrac{\partial(x, y)}{\partial(u, v)}\right|\ddi u\ddi v.$$
\end{theorem}

\begin{remark}
    Notice the correspondence between Theorem \ref*{cv di} and the substitution rule in one-dimensional integrals: $$\displaystyle\int_a^b f(x)\ddi x=\int_{\alpha}^{\beta}f(\phi(t))\derivative{\phi}{t}(t)\ddi t.$$
\end{remark}


\begin{example}
    Use the change of variables $\left\{\begin{array}{l}
        x=u^2-v^2\\
        y=2uv
    \end{array}\right.$ to evaluate the integral $\displaystyle\iint_R y\ddi A,$ where $R$ is the region bounded by the $x$-axis and the parabolas $y^2=4-4x$, and $y^2=4+4x$ with $y\geq 0$.
\end{example}


\begin{example}
    Use the transformation $\left\{\begin{array}{l}
        u=x-y\\
        v=x+y
    \end{array}\right.$ to evaluate $\displaystyle\iint_R\dfrac{x-y}{x+y}\ddi A,$ where $R$ is the square with vertices $(0, 2)$, $(1, 1)$, $(2, 2)$, and $(1, 3)$.
\end{example}


\begin{example}
    Evaluate the integral $\displaystyle\iint_R e^{\frac{x+y}{x-y}}\ddi A,$ where $R$ is the trapezoidal region with vertices $(1, 0)$, $(2, 0)$, $(0, -2)$, and $(0, -1)$.
\end{example}

\end{document}