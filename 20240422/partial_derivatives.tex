% LaTeX Template For MATH 490 @ VCU
\documentclass[11pt]{article}

\usepackage{hyperref}
\usepackage{amsmath}
\usepackage{amsthm}
\usepackage{amssymb}
\usepackage{enumerate}
\usepackage{enumitem}
\usepackage{titlesec}
\usepackage{multicol}
\usepackage{multirow}
\usepackage{mathtools}
\usepackage{mdframed}
\usepackage{tocloft}
\usepackage{tcolorbox}
\usepackage{extarrows}

\setlist{nosep}
% \setlist[enumerate]{label=(\alph*)}

\renewcommand{\arraystretch}{0.75}

\definecolor{defcolor}{RGB}{255,236,236}    % light red
\definecolor{ngtcolor}{RGB}{255,242,242}    % lighter red
\definecolor{lnkcolor}{RGB}{0,0,180}        % blue
\definecolor{thmcolor}{RGB}{236,236,255}    % light blue
\definecolor{lemcolor}{RGB}{239,239,255}    % lighter blue
\definecolor{procolor}{RGB}{242,242,255}    % lighter lighter blue
\definecolor{crlcolor}{RGB}{245,245,255}    % lighter lighter lighter blue
\definecolor{xmpcolor}{RGB}{255,240,225}    % light orange
\definecolor{rmkcolor}{RGB}{233,255,235}    % light green
\definecolor{axicolor}{RGB}{255,255,233}    % light yellow
\definecolor{notcolor}{RGB}{255,255,244}    % lighter yellow
\definecolor{whacolor}{RGB}{250,250,250}    % lighter gray
\definecolor{reccolor}{RGB}{255,244,255}    % lighter purple

\hypersetup{
    colorlinks,
    citecolor=lnkcolor,
    filecolor=lnkcolor,
    linkcolor=lnkcolor,
    urlcolor=lnkcolor
}

\newtheoremstyle{break}
    {\topsep/1.5} % space above
    {\topsep/2.2} % space below
    {}          % body font
    {}          % indent amount
    {\rmfamily} % theorem head font
    {.}          % punctuation after theorem head
    {\newline}  % space after theorem head
    {\textbf{\thmname{#1}\thmnumber{ #2}}\thmnote{\text{ (#3)}}}
                % theorem hed spec. (empty = "normal")

\theoremstyle{break}
\newmdtheoremenv[
    backgroundcolor=thmcolor,
    linecolor=black,
    linewidth=1pt,
    topline=true,
    bottomline=true,
    rightline=true,
    skipabove=\topsep/1.5,
    skipbelow=\topsep/2.2
]{theorem}{Theorem}[section]
\newmdtheoremenv[
    backgroundcolor=crlcolor,
    linecolor=black,
    linewidth=1pt,
    topline=true,
    bottomline=true,
    rightline=true,
    skipabove=\topsep/1.5,
    skipbelow=\topsep/2.2
]{corollary}[theorem]{Corollary}
\newmdtheoremenv[
    backgroundcolor=lemcolor,
    linecolor=black,
    linewidth=1pt,
    topline=true,
    bottomline=true,
    rightline=true,
    skipabove=\topsep/1.5,
    skipbelow=\topsep/2.2
]{lemma}[theorem]{Lemma}
\newmdtheoremenv[
    backgroundcolor=axicolor,
    linecolor=black,
    linewidth=1pt,
    topline=true,
    bottomline=true,
    rightline=true,
    skipabove=\topsep/1.5,
    skipbelow=\topsep/2.2
]{axiom}[theorem]{Axiom}
\newmdtheoremenv[
    backgroundcolor=procolor,
    linecolor=black,
    linewidth=1pt,
    topline=true,
    bottomline=true,
    rightline=true,
    skipabove=\topsep/1.5,
    skipbelow=\topsep/2.2
]{proposition}[theorem]{Proposition}
\newmdtheoremenv[
    backgroundcolor=notcolor,
    linecolor=black,
    linewidth=1pt,
    topline=true,
    bottomline=true,
    rightline=true,
    skipabove=\topsep/1.5,
    skipbelow=\topsep/2.2
]{notation}[theorem]{Notation}
\newmdtheoremenv[
    backgroundcolor=defcolor,
    linecolor=black,
    linewidth=1pt,
    topline=true,
    bottomline=true,
    rightline=true,
    skipabove=\topsep/1.5,
    skipbelow=\topsep/2.2
]{definition}[theorem]{Definition}
\newmdtheoremenv[
    backgroundcolor=ngtcolor,
    linecolor=black,
    linewidth=1pt,
    topline=true,
    bottomline=true,
    rightline=true,
    skipabove=\topsep/1.5,
    skipbelow=\topsep/2.2
]{negation}[theorem]{Negation}
\newmdtheoremenv[
    backgroundcolor=rmkcolor,
    linecolor=black,
    linewidth=1pt,
    topline=true,
    bottomline=true,
    rightline=true,
    skipabove=\topsep/1.5,
    skipbelow=\topsep/2.2
]{remark}[theorem]{Remark}
\newmdtheoremenv[
    backgroundcolor=xmpcolor,
    linecolor=black,
    linewidth=1pt,
    topline=true,
    bottomline=true,
    rightline=true,
    skipabove=\topsep/1.5,
    skipbelow=\topsep/2.2
]{example}[theorem]{Example}
\newmdtheoremenv[
    backgroundcolor=whacolor,
    linecolor=black,
    linewidth=1pt,
    topline=true,
    bottomline=true,
    rightline=true,
    skipabove=\topsep/1.5,
    skipbelow=\topsep/2.2
]{problem}[theorem]{Problem}
\newmdtheoremenv[
    backgroundcolor=whacolor,
    linecolor=black,
    linewidth=1pt,
    topline=true,
    bottomline=true,
    rightline=true,
    skipabove=\topsep/1.5,
    skipbelow=\topsep/2.2
]{question}[theorem]{Question}
\newmdtheoremenv[
    backgroundcolor=reccolor,
    linecolor=black,
    linewidth=1pt,
    topline=true,
    bottomline=true,
    rightline=true,
    skipabove=\topsep/1.5,
    skipbelow=\topsep/2.2
]{recall}[theorem]{Recall}

\DeclareMathOperator{\arcsec}{arcsec}
\DeclareMathOperator{\arccot}{arccot}
\DeclareMathOperator{\arccsc}{arccsc}
\DeclareMathOperator{\interior}{int}
\DeclareMathOperator{\closure}{cl}
\DeclareMathOperator{\boundary}{bd}

\newcommand{\differentiate}[1]{\dfrac{\dd}{\dd{#1}}}
\newcommand{\pdifferentiate}[1]{\dfrac{\partial}{\partial {#1}}}
\newcommand{\derivative}[2]{\dfrac{\dd{#1}}{\dd{#2}}}
\newcommand{\scndderivative}{D^2\!\,}
\newcommand{\highderivative}[1]{D^{#1}\!\,}
\newcommand{\dirderivative}[1]{D_{#1}\!\,}
\newcommand{\pderivative}[2]{\dfrac{\partial {#1}}{\partial {#2}}}
\newcommand{\scndpderivative}[3]{\dfrac{\partial^2 {#1}}{\partial {#3}\partial {#2}}}
\newcommand{\highpderivative}[4]{\dfrac{\partial^{#2} {#1}}{\partial{#4}\cdots\partial{#3}}}
\newcommand{\dd}{\text{d}}
\newcommand{\ddi}{\text{$\,$d}}
\newcommand{\qqed}{{\hfill$\blacksquare$}}
\newcommand{\defeq}{\overset{\text{def}}{=}}
\newcommand{\transpose}{\text{T}}
\newcommand{\bbR}{\mathbb{R}}
\newcommand{\bbN}{\mathbb{N}}
\newcommand{\calL}{\mathcal{L}}
\newcommand{\bfa}{\textbf{a}}
\newcommand{\bfc}{\textbf{c}}
\newcommand{\bfe}{\textbf{e}}
\newcommand{\bff}{\textbf{f}}
\newcommand{\bfg}{\textbf{g}}
\newcommand{\bfh}{\textbf{h}}
\newcommand{\bfp}{\textbf{p}}
\newcommand{\bfr}{\textbf{r}}
\newcommand{\bfv}{\textbf{v}}
\newcommand{\bfu}{\textbf{u}}
\newcommand{\bfx}{\textbf{x}}
\newcommand{\bfy}{\textbf{y}}
\newcommand{\exercise}{This is an exercise left to the reader.}


\linespread{1.9}
\setlength{\textwidth}{6.9in}
\setlength{\textheight}{9.2in}
\setlength{\oddsidemargin}{-0.2in}
\setlength{\evensidemargin}{-0.2in}
\setlength{\topmargin}{-0.2in}
\setlength{\headheight}{0in}
\setlength{\headsep}{0in}
\setlength{\footskip}{0.5in}
\setlength{\multicolsep}{6.2pt}
\setlength{\belowdisplayskip}{0pt}
%\setlength{\belowdisplayshortskip}{0pt}
\setlength{\abovedisplayskip}{0pt}
%\setlength{\abovedisplayshortskip}{0pt}

\setcounter{section}{3}
\numberwithin{equation}{theorem}

\makeatletter
\newcommand{\vast}{\bBigg@{4}}
\newcommand{\Vast}{\bBigg@{5}}
\makeatother

\newcommand*\samethanks[1][\value{footnote}]{\footnotemark[#1]}

\title{\textbf{Calculus A II One-to-One Tutoring}}
\author{Chang, Yung-Hsuan}

\begin{document}
\maketitle

\begin{example}
    Compute all the first partial derivatives of $f(x, y)=x^3+x^2y^2-2y^2$.
\end{example}
\vspace{8em}

\begin{recall}[Chain Rule]
    If $g$ is differentiable at $x$ and $f$ is differentiable at $g(x)$, then $(f\circ g)(x)$ is differentiable at $x$ and $$\differentiate{x}\left((f\circ g)(x)\right)=\derivative{f}{x}(g(x))\cdot\derivative{g}{x}(x).$$
    In Leibniz notation, if $y=f(u)$ and $u=g(x)$ are both differentiable, then $$\derivative{y}{x}=\derivative{y}{u}\cdot\derivative{u}{x}.$$
\end{recall}

\begin{theorem}[Chain Rule]
    Let $z=f(x, y), x=x(t), y=y(t)$. Then $$\derivative{z}{t}=\pderivative{z}{x}\cdot\derivative{x}{t}+\pderivative{z}{y}\cdot\derivative{y}{t}.$$
\end{theorem}

\begin{example}
    Let $z=x^2y+3xy^4$ with $(x, y)=(\sin(2t), \cos(t))$. Compute $\derivative{z}{t}$.
\end{example}
\vspace{15em}

\begin{example}
    Let $u=x^4y+y^2z^3$ with $(x, y, z)=(rse^t, rs^2e^{-t}, r^2s\sin(t))$. Compute $\pderivative{u}{s}$.
\end{example}
\vspace{15em}

\begin{example}
    Compute all first partial derivatives for $f(x, y)=e^{\sqrt{x^2+y}}$.
\end{example}
\vspace{15em}

\begin{example}
    Let $z=x^2+y^2$ with $(x, y)=(t-\cos t, 1-\sin t)$. Compute $\derivative{z}{t}$.
\end{example}
\vspace{15em}

\begin{example}
    Compute all first partial derivatives for $f(x, y)=\arcsin\left(\dfrac{x^2-y^2}{x^2+y^2}\right)$ with $xy>0$.
\end{example}
\newpage

\begin{example}
    Let $z=e^{xy}$ with $(x, y)=\left(\ln\sqrt{u^2+v^2}, \arctan\dfrac{v}{u}\right)$. Compute $\pderivative{z}{u}$.
\end{example}
\newpage

\begin{definition}[Directional Derivative]
    Let $f$ be a real-valued function of two variables. The directional derivative of $f$ at $(x_0, y_0)$ for the direction $\hat{\bfu}= (u_1, u_2)$ (a unit vector) is defined when the following limit exists:
    $$\dirderivative{\hat{\bfu}}f(x_0, y_0)=\lim_{h\to 0}\dfrac{f(x_0+hu_1, y_0+hu_2)-f(x_0, y_0)}{h}.$$
\end{definition}

\begin{remark}
    In the previous definition, if $\hat{\bfu}=(1, 0)$, then $$\dirderivative{\hat{\bfu}}f(x_0, y_0)=\pderivative{f}{x}(x_0, y_0).$$
\end{remark}

\begin{example}
    Determine the directional derivative of $$f(x, y)=\left\{\begin{array}{ll}
        \dfrac{xy}{x^2+y^2}, \quad&\text{if\ }(x, y)\ne(0, 0);\\
        0, &\text{if\ }(x, y)=(0, 0),
    \end{array}\right.$$
    for the direction $(1, 1)$ at $(0, 0)$.
\end{example}
\vspace{20em}

\begin{example}
    Determine the directional derivative of $f(x, y)=xy^2\arctan(z)$ for the direction $(1, 1, 1)$ at $(2, 1, 1)$.
\end{example}
\vspace{20em}

\begin{definition}[Limit]
    The limit $$\lim_{(x, y)\to(x_0, y_0)}f(x, y)=L$$ if for all $\varepsilon>0$ (the greek letter $\varepsilon$ stands for ``error''), there exists a $\delta>0$ (the greek letter $\delta$ stands for ``difference'') such that $\sqrt{(x-x_0)^2+(y-y_0)^2}<\delta$ implies $|f(x, y)-f(x_0, y_0)|<\varepsilon$.
\end{definition}

\begin{remark}
    If there exist two path $C_1: y=f_1(x)$ and $C_2: y=f_2(x)$ such that $$\lim_{\substack{(x, y)\to(x_0, y_0)\\(x, y)\in C_1}}f(x, y)\ne\lim_{\substack{(x, y)\to(x_0, y_0)\\(x, y)\in C_2}}f(x, y),$$ then the limit $\displaystyle\lim_{(x, y)\to(x_0, y_0)}f(x, y)$ does not exist.
\end{remark}

\begin{example}
    Determine the existance of $\displaystyle\lim_{(x, y)\to(0, 0)}f(x, y)$ for each of the following functions:
    \begin{enumerate}
        \item $f(x, y)=\dfrac{x^2-y^2}{x^2+y^2}$;
        \item $f(x, y)=\dfrac{xy}{x^2+y^2}$;
        \item $f(x, y)=\dfrac{x^2y}{x^2+y^2}$; and
        \item $f(x, y)=\dfrac{x^\frac{2}{3}\sin y}{x^2+y^2}$.
    \end{enumerate}
\end{example}
\vspace{15em}

\begin{example}
    Determine $\displaystyle\lim_{(x, y)\to(0, 0)}\dfrac{\sin(x^2+y^2)}{x^2+y^2}$.
\end{example}
\vspace{8em}

\begin{definition}[Continuity]
    We say a function $f$ is continuous at $(x_0, y_0)$ if $$\lim_{(x, y)\to(x_0, y_0)}f(x, y)=f(x_0, y_0).$$ We say a function is continuous if it is continuous at every point in its domain.
\end{definition}

\begin{example}
    Determine the conitnuity of the function $f(x, y)=\dfrac{x^2y}{x^2+y^2}$ at $(0, 0)$.
\end{example}



\end{document}