% LaTeX Template For MATH 490 @ VCU
\documentclass[11pt]{article}

\usepackage{hyperref}
\usepackage{amsmath}
\usepackage{amsthm}
\usepackage{amssymb}
\usepackage{enumerate}
\usepackage{enumitem}
\usepackage{titlesec}
\usepackage{multicol}
\usepackage{multirow}
\usepackage{mathtools}
\usepackage{mdframed}
\usepackage{tocloft}
\usepackage{tcolorbox}
\usepackage{extarrows}

\setlist{nosep}
% \setlist[enumerate]{label=(\alph*)}

\renewcommand{\arraystretch}{0.75}

\definecolor{defcolor}{RGB}{255,236,236}    % light red
\definecolor{ngtcolor}{RGB}{255,242,242}    % lighter red
\definecolor{lnkcolor}{RGB}{0,0,180}        % blue
\definecolor{thmcolor}{RGB}{236,236,255}    % light blue
\definecolor{lemcolor}{RGB}{239,239,255}    % lighter blue
\definecolor{procolor}{RGB}{242,242,255}    % lighter lighter blue
\definecolor{crlcolor}{RGB}{245,245,255}    % lighter lighter lighter blue
\definecolor{xmpcolor}{RGB}{255,240,225}    % light orange
\definecolor{rmkcolor}{RGB}{233,255,235}    % light green
\definecolor{axicolor}{RGB}{255,255,233}    % light yellow
\definecolor{notcolor}{RGB}{255,255,244}    % lighter yellow
\definecolor{whacolor}{RGB}{250,250,250}    % lighter gray
\definecolor{reccolor}{RGB}{255,244,255}    % lighter purple

\hypersetup{
    colorlinks,
    citecolor=lnkcolor,
    filecolor=lnkcolor,
    linkcolor=lnkcolor,
    urlcolor=lnkcolor
}

\newtheoremstyle{break}
    {\topsep/1.5} % space above
    {\topsep/2.2} % space below
    {}          % body font
    {}          % indent amount
    {\rmfamily} % theorem head font
    {.}          % punctuation after theorem head
    {\newline}  % space after theorem head
    {\textbf{\thmname{#1}\thmnumber{ #2}}\thmnote{\text{ (#3)}}}
                % theorem hed spec. (empty = "normal")

\theoremstyle{break}
\newmdtheoremenv[
    backgroundcolor=thmcolor,
    linecolor=black,
    linewidth=1pt,
    topline=true,
    bottomline=true,
    rightline=true,
    skipabove=\topsep/1.5,
    skipbelow=\topsep/2.2
]{theorem}{Theorem}[section]
\newmdtheoremenv[
    backgroundcolor=crlcolor,
    linecolor=black,
    linewidth=1pt,
    topline=true,
    bottomline=true,
    rightline=true,
    skipabove=\topsep/1.5,
    skipbelow=\topsep/2.2
]{corollary}[theorem]{Corollary}
\newmdtheoremenv[
    backgroundcolor=lemcolor,
    linecolor=black,
    linewidth=1pt,
    topline=true,
    bottomline=true,
    rightline=true,
    skipabove=\topsep/1.5,
    skipbelow=\topsep/2.2
]{lemma}[theorem]{Lemma}
\newmdtheoremenv[
    backgroundcolor=axicolor,
    linecolor=black,
    linewidth=1pt,
    topline=true,
    bottomline=true,
    rightline=true,
    skipabove=\topsep/1.5,
    skipbelow=\topsep/2.2
]{axiom}[theorem]{Axiom}
\newmdtheoremenv[
    backgroundcolor=procolor,
    linecolor=black,
    linewidth=1pt,
    topline=true,
    bottomline=true,
    rightline=true,
    skipabove=\topsep/1.5,
    skipbelow=\topsep/2.2
]{proposition}[theorem]{Proposition}
\newmdtheoremenv[
    backgroundcolor=notcolor,
    linecolor=black,
    linewidth=1pt,
    topline=true,
    bottomline=true,
    rightline=true,
    skipabove=\topsep/1.5,
    skipbelow=\topsep/2.2
]{notation}[theorem]{Notation}
\newmdtheoremenv[
    backgroundcolor=defcolor,
    linecolor=black,
    linewidth=1pt,
    topline=true,
    bottomline=true,
    rightline=true,
    skipabove=\topsep/1.5,
    skipbelow=\topsep/2.2
]{definition}[theorem]{Definition}
\newmdtheoremenv[
    backgroundcolor=ngtcolor,
    linecolor=black,
    linewidth=1pt,
    topline=true,
    bottomline=true,
    rightline=true,
    skipabove=\topsep/1.5,
    skipbelow=\topsep/2.2
]{negation}[theorem]{Negation}
\newmdtheoremenv[
    backgroundcolor=rmkcolor,
    linecolor=black,
    linewidth=1pt,
    topline=true,
    bottomline=true,
    rightline=true,
    skipabove=\topsep/1.5,
    skipbelow=\topsep/2.2
]{remark}[theorem]{Remark}
\newmdtheoremenv[
    backgroundcolor=xmpcolor,
    linecolor=black,
    linewidth=1pt,
    topline=true,
    bottomline=true,
    rightline=true,
    skipabove=\topsep/1.5,
    skipbelow=\topsep/2.2
]{example}[theorem]{Example}
\newmdtheoremenv[
    backgroundcolor=whacolor,
    linecolor=black,
    linewidth=1pt,
    topline=true,
    bottomline=true,
    rightline=true,
    skipabove=\topsep/1.5,
    skipbelow=\topsep/2.2
]{problem}[theorem]{Problem}
\newmdtheoremenv[
    backgroundcolor=whacolor,
    linecolor=black,
    linewidth=1pt,
    topline=true,
    bottomline=true,
    rightline=true,
    skipabove=\topsep/1.5,
    skipbelow=\topsep/2.2
]{question}[theorem]{Question}
\newmdtheoremenv[
    backgroundcolor=reccolor,
    linecolor=black,
    linewidth=1pt,
    topline=true,
    bottomline=true,
    rightline=true,
    skipabove=\topsep/1.5,
    skipbelow=\topsep/2.2
]{recall}[theorem]{Recall}

\DeclareMathOperator{\arcsec}{arcsec}
\DeclareMathOperator{\arccot}{arccot}
\DeclareMathOperator{\arccsc}{arccsc}
\DeclareMathOperator{\interior}{int}
\DeclareMathOperator{\closure}{cl}
\DeclareMathOperator{\boundary}{bd}

\newcommand{\differentiate}[1]{\dfrac{\dd}{\dd{#1}}}
\newcommand{\derivative}[2]{\dfrac{\dd{#1}}{\dd{#2}}}
\newcommand{\scndderivative}{D^2\!\,}
\newcommand{\highderivative}[1]{D^{#1}\!\,}
\newcommand{\dirderivative}[1]{D_{#1}\!\,}
\newcommand{\pderivative}[2]{\dfrac{\partial {#1}}{\partial {#2}}}
\newcommand{\scndpderivative}[3]{\dfrac{\partial^2 {#1}}{\partial {#3}\partial {#2}}}
\newcommand{\highpderivative}[4]{\dfrac{\partial^{#2} {#1}}{\partial{#4}\cdots\partial{#3}}}
\newcommand{\dd}{\text{d}}
\newcommand{\ddi}{\text{$\,$d}}
\newcommand{\qqed}{{\hfill$\blacksquare$}}
\newcommand{\defeq}{\overset{\text{def}}{=}}
\newcommand{\transpose}{\text{T}}
\newcommand{\bbR}{\mathbb{R}}
\newcommand{\bbN}{\mathbb{N}}
\newcommand{\calL}{\mathcal{L}}
\newcommand{\bfa}{\textbf{a}}
\newcommand{\bfc}{\textbf{c}}
\newcommand{\bfe}{\textbf{e}}
\newcommand{\bff}{\textbf{f}}
\newcommand{\bfg}{\textbf{g}}
\newcommand{\bfh}{\textbf{h}}
\newcommand{\bfp}{\textbf{p}}
\newcommand{\bfr}{\textbf{r}}
\newcommand{\bfv}{\textbf{v}}
\newcommand{\bfu}{\textbf{u}}
\newcommand{\bfx}{\textbf{x}}
\newcommand{\bfy}{\textbf{y}}
\newcommand{\exercise}{This is an exercise left to the reader.}


\linespread{1.9}
\setlength{\textwidth}{6.9in}
\setlength{\textheight}{9.2in}
\setlength{\oddsidemargin}{-0.2in}
\setlength{\evensidemargin}{-0.2in}
\setlength{\topmargin}{-0.2in}
\setlength{\headheight}{0in}
\setlength{\headsep}{0in}
\setlength{\footskip}{0.5in}
\setlength{\multicolsep}{6.2pt}
\setlength{\belowdisplayskip}{0pt}
%\setlength{\belowdisplayshortskip}{0pt}
\setlength{\abovedisplayskip}{0pt}
%\setlength{\abovedisplayshortskip}{0pt}

\setcounter{section}{1}
\numberwithin{equation}{theorem}

\makeatletter
\newcommand{\vast}{\bBigg@{4}}
\newcommand{\Vast}{\bBigg@{5}}
\makeatother

\newcommand*\samethanks[1][\value{footnote}]{\footnotemark[#1]}

\title{\textbf{Calculus A II One-to-One Tutoring}}
\author{Chang, Yung-Hsuan}

\begin{document}
\maketitle

\begin{question}
    Evaluate $\displaystyle\sum_{i=1}^\infty\left(\dfrac{1}{2}\right)^i$.
\end{question}

\begin{theorem}[Formulae about Derivatives]
    The symbol $x$ denotes a variable. The symbols $a, r$ are constant and are in $\bbR^+, \bbR$, respectively. \vspace{-1.8em}
    \begin{multicols}{2}
        \begin{enumerate}
            \item $\differentiate{x}\left(x^r\right)=rx^{r-1}$;
            \item $\differentiate{x}\left(a^x\right)=\ln a\cdot a^x$;
            \item $\differentiate{x}\left(\ln x\right)=\dfrac{1}{x}$;
            \item $\differentiate{x}(\sin x)=\cos x$; and
            \item $\differentiate{x}(\cos x)=-\sin x$.
        \end{enumerate}
    \end{multicols}
    \vspace{0.2em}
\end{theorem}

\begin{theorem}[Operations on Derivatives]
    Let $f(x)$ and $g(x)$ be differentiable. Then
    \begin{enumerate}
        \item $\differentiate{x}\left(f(x)+g(x)\right)=\derivative{f}{x}(x)+\derivative{g}{x}(x)$;
        \item $\differentiate{x}\left(c\cdot f(x)\right)=c\cdot\derivative{f}{x}(x)$;
        \item $\differentiate{x}\left(f(x)\cdot g(x)\right)=\derivative{f}{x}(x)\cdot g(x)+f(x)\cdot\derivative{g}{x}(x)$; and 
        \item $\differentiate{x}\left(\dfrac{f(x)}{g(x)}\right)=\dfrac{\derivative{f}{x}(x)\cdot g(x)-f(x)\cdot\derivative{g}{x}(x)}{\left(g(x)\right)^2}$.
    \end{enumerate}
\end{theorem}

\begin{example}
    Find the derivative of $x^4$ with respect to $x$.
\end{example}
\vspace{8em}

\begin{example}
    Find the derivative of $t^{100}$ with respect to $t$.
\end{example}
\vspace{8em}

\begin{example}
    Find the derivative of $\dfrac{1}{u}$ with respect to $u$.
\end{example}
\vspace{8em}

\begin{example}
    Find the derivative of $x\sqrt{x}$ with respect to $x$.
\end{example}
\vspace{8em}

\begin{example}
    Find the derivative of $x^4-6x^2+4$ with respect to $x$.
\end{example}
\vspace{8em}

\begin{example}
    Find the derivative of $e^t+t^e$ with respect to $t$.
\end{example}
\vspace{8em}

\begin{example}
    Find the derivative of $x^{1000}$ with respect to $x$.
\end{example}
\vspace{8em}

\begin{example}
    Evaluate $\displaystyle\lim_{x\to 1}\dfrac{x^{1000}-1}{x-1}$.
\end{example}
\vspace{8em}

\begin{example}
    Find the derivative of $\dfrac{x^2+x-2}{x^3+6}$ with respect to $x$.
\end{example}
\vspace{8em}

\begin{example}
    Find the derivative of $\dfrac{e^x}{1-e^x}$ with respect to $x$.
\end{example}
\vspace{8em}

\begin{example}
    Find the derivative of $x^3e^x$ with respect to $x$.
\end{example}
\vspace{8em}

\begin{example}
    Find the derivative of $\sin\theta+\cos\theta$ with respect to $\theta$.
\end{example}
\vspace{8em}

\begin{example}
    Find the derivative of $\sin\theta\cdot\cos\theta$ with respect to $\theta$.
\end{example}
\vspace{8em}

\begin{example}
    Find the derivative of $\dfrac{\sin x}{\cos x}$ with respect to $x$.
\end{example}
\vspace{8em}

\begin{example}
    Find the derivative of $\dfrac{\cos\theta}{e^\theta}$ with respect to $\theta$.
\end{example}
\vspace{8em}

\newpage
\begin{question}
    Find the derivative with respect to the corresponding variable for the following functions:
    \begin{enumerate}
        \item $f_1(x)=\pi^{25}$;
        \item $f_2(x)=(4x^2+3)(2x+5)$;
        \item $f_3(x)=\dfrac{x}{e^x}$;
        \item $f_4(u)=\dfrac{6u^4-5u}{u+1}$;
        \item $f_5(t)=\dfrac{t^4-5t^3+\sqrt{t}}{t^2}$;
        \item $f_6(\theta)=\tan(\theta)$;
        \item $f_7(\theta)=\sec(\theta)$; and
        \item $f_8(x)=e^x\cos x+\sin x$.
    \end{enumerate}
\end{question}
\newpage

\begin{theorem}[Chain Rule]
    If $g$ is differentiable at $x$ and $f$ is differentiable at $g(x)$, then $(f\circ g)(x)$ is differentiable at $x$ and $$\differentiate{x}\left((f\circ g)(x)\right)=\derivative{f}{x}(g(x))\cdot\derivative{g}{x}(x).$$
    In Leibniz notation, if $y=f(u)$ and $u=g(x)$ are both differentiable, then $$\derivative{y}{x}=\derivative{y}{u}\cdot\dfrac{u}{x}.$$
\end{theorem}

\begin{example}
    Find the derivative of $\left(x^3-1\right)^{100}$ with respect to $x$.
\end{example}
\vspace{8em}

\begin{example}
    Find the derivative of $\left(\dfrac{t-1}{2t+1}\right)^{9}$ with respect to $t$.
\end{example}
\vspace{8em}

\begin{example}
    Find the derivative of $\dfrac{1}{\sin u}$ with respect to $u$.
\end{example}
\vspace{8em}

\begin{example}
    Find the derivative of $x^4-6x^2+4$, where $x=\sin v$, with respect to $v$.
\end{example}
\vspace{8em}

\begin{example}
    Find the derivative of $\left(e^{\sin x}\right)^{100}$ with respect to $x$.
\end{example}
\vspace{8em}

\begin{example}
    Find the derivative of $e^{e^x}$ with respect to $x$.
\end{example}
\vspace{8em}

\begin{example}
    Find the derivative of $\sin(\cos x)$ with respect to $x$.
\end{example}
\vspace{8em}

\begin{example}
    Find the derivative of $\sin\left(\dfrac{e^x}{1-e^x}\right)$ with respect to $x$.
\end{example}
\vspace{8em}

\newpage
\begin{question}
    Find the derivative with respect to the corresponding variable for the following functions:
    \begin{enumerate}
        \item $f_1(x)=\left(x^2+1\right)^{25}$;
        \item $f_2(x)=\cos(x^2)$;
        \item $f_3(x)=\left(\cos x\right)^2$;
        \item $f_4(u)=e^{u^2-u}$;
        \item $f_5(t)=e^{at}\sin(bt)$;
        \item $f_6(\theta)=\cos(\sin(3\theta))$;
        \item $f_7(\theta)=\sqrt{\theta+\sqrt{\theta}}$;
        \item $f_8(x)=5^{\sqrt{x}+\sin(x)}$;
        \item $f_9(x)=\ln(\sin x)$; and
        \item $f_{10}(x)=e^{x^2\ln x}$.
    \end{enumerate}
\end{question}
\newpage
\ 
\newpage


\end{document}